\documentclass[letterpaper,twocolumn,10pt]{article}
\usepackage{usenix2019_v3}

% to be able to draw some self-contained figs
\usepackage{tikz}
\usepackage{amsmath}

\begin{document}

%don't print date
\date{}

% make title bold and 14 pt font (Latex default is non-bold, 16 pt)
\title{\Large \bf Dark Decoys:
  Conjuring Proxies from Unused Address Space}

%for single author (just remove % characters)
%\author{
%{\rm Your N.\ Here}\\
%Your Institution
%\and
%{\rm Second Name}\\
%Second Institution
%% copy the following lines to add more authors
%% \and
%% {\rm Name}\\
%%Name Institution
%} % end author

\maketitle

%-------------------------------------------------------------------------------
\begin{abstract}
%-------------------------------------------------------------------------------
Refraction Networking (formerly ``Decoy Routing'') has emrged as a useful tool in circumventing
        Internet censorship. By placing proxies in cooperating Internet Service
        Providers (ISPs) and using connections to existing reachable
        ``decoy'' sites for transport, censors cannot easily block access to
        such proxies without also blocking legitimate sites.

However, existing deployed Refraction Networking schemes such as TapDance suffer
from several problems, including a limited number of decoy sites in realistic
deployments, and an unfavorable tradeoff between performance and observability
by the censor. These problems limit where such proxies can be deloyed, hamper
their effectiveness, and may ultimately make them possible for censors to block.

In this paper, we present Dark Decoys, a deployable Refraction Networking scheme that
overcomes these issues. Dark Decoys leverages the (primarily) unused address
space that is reachable through the deploying ISP. Rather than rely on existing
sites as reachable decoy sites that must be involved in each proxy connection,
Dark Decoys create seemingly legitimate hosts at new IP addresses. These
invented hosts are indistinguishable from normal hosts to the censor, but can
be used by clients as one-time proxies.

We implement our scheme ... and it's just the bee's knees. TODO

\end{abstract}


%-------------------------------------------------------------------------------
\section{Introduction}
%-------------------------------------------------------------------------------
% 1.5-2 pages

Censorship circumvention still important
-More countries blocking
-Existing countries blocking more

Existing circumvention strategies on shaky ground
-domain fronting going away
-active probing of existing proxies
-Cenosrs fingerprinting known protocols

Importance of Refraction Networking
-what it is, how it solves many of the above problems
-acknowledge challenge of ISP deployment, cite TapDance deployment as only refraction technology that has overcome this challenge so far

Challenges remain in Tapdance:
-Censor can fingerprint decoy sites
-Decoys themselves are limited (e.g. a few dozen in many cases)
-Can't have long-lived connections (performance and observability issue)
-Other performance limits (upload limit, TCP window size)

Dark decoys solve these issues
-Create new decoys from ``dark'' (unused) address space
-Clients register (via TapDance-like or other robust protocol (email, blockchain, whatever))
-Connect to custom IP address, talk whatever protocol client/station agrees on

One-time use address advantages:
-Censor cannot actively probe ahead of time (especially in IPv6), making it hard to fingerprint
-Decoys are now virtually unlimited (or limited only by address space)
-Connection can live as long as we want
-No pesky TCP/TapDance-y limits

Requirements/Challenges
-Censor can't be able to distinguish between dark decoy and legitimate hosts
    (otherwise they block all dark decoys)
    -Includes active probing: censor shouldn't be able to register existing IP
-Disruption avoidence: Want to pickup even for used addresses
    -Otherwise, censor probes to find truly unused address space and blocks
    -But can't disrupt legitimate services
    -Can overcome by limiting pickup to the registering client (but spoofing challenges...)
    -Note: also likely solved by the client-sends-SNI in the Mask site application...



\section{Background}
% 1 page
Background on TLS and Refraction Networking


\section{Threat Model}
% 0.5 page

-Censor can block arbitrary addresses (or networks), but faces a cost in doing so (collateral damage)
-Censor can know what network deploys the dark decoy station
-Censor knows the dark decoy prefixes distributed in the client, but they contain legitimate hosts
-Censor can use the client
-Censor can active probe limited sets, but cannot enumerate the entire prefix (i.e. IPv6 /32)
-Censor can active probe or (p)replay connections it suspects


\section{Architecture}
% 3 pages
-Registration
-Dark decoy selection (seed -> IP)
-Connection to Dark Decoy Application (DD App)
-IPv6 vs IPv4 dark decoys

\subsection{Applications}

Application choices:
-obfs4  (could we test this?)
-Masked sites (implemented)
-TLS 1.3 eSNI

-Takeover of masked site TLS connection (client/server random from masked site, master secret from registration)

-Masked site slection: How do we get the list of masked sites? Alexa list? passive observation?

\section{Implementation}
% 1 page

-ISP-scale implementation, able to handle 10+Gbps
-Language choice (Rust, Go, minimal C)
-Open source (client and station)

\section{Evaluation}
% 1 page

Compare to TapDance performance:
-Connection setup time
-Connection/session lifetime
-Throughput (especially upload vs download)
-Security considerations

\section{Attacks and Defenses}
% 1 page

-Fingerprinting Masked site vs Dark decoy application
-Active probing TLS connection
    -Block access unless from registering source IP
        -cite GFW taking over client IPs for active probing of old obfs
    -Verify knowledge of seed (but how to respond when verification fails?)

\section{Related Work}
% 0.5-0.75 page

MultiFlow, The Waterfall of Liberty, Slitheen, Rebound, TapDance, Cirripede, Telex, Decoy Routing
True Cost of RAD, Routing around Decoys
ISP-scale TapDance


\section{Future Work}
% 0.5 page

\section{Conclusion}
%.25 page

%%-------------------------------------------------------------------------------
%\section*{Availability}
%%-------------------------------------------------------------------------------
%
%USENIX program committees give extra points to submissions that are
%backed by artifacts that are publicly available. If you made your code
%or data available, it's worth mentioning this fact in a dedicated
%section.

%-------------------------------------------------------------------------------
\bibliographystyle{plain}
\bibliography{biblio}

\end{document}
